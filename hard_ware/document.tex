\documentclass[UTF8]{ctexart}
\usepackage{amsmath}
\usepackage{mathrsfs}
\usepackage{geometry}
\usepackage{circuitikz}
\geometry{left=2.0cm, right=2.0cm, top=2.0cm, bottom=2.5cm}
\title{模拟电路基础}
%\author{QuantumBird}
\date{\today}
\begin{document}
	\section{三极管}
		\subsubsection{失真}
			\paragraph{截止失真}
				Q点过低
			\paragraph{饱和失真}
				Q点过高
			\paragraph{最大不失真幅度}
		\paragraph{动态失真}
			\paragraph{有负载$R_L$}
			\begin{itemize}\title{方法}
				\item 画出交流等效电路:
					\textbf{交流等效电路}中
						电容相当于短路, 电源相当于接地\par
					画直流通路时管子站着画,画交流通路共基级时躺着画;输入信号从左边进入。
				\item 过Q点$(V_{CEQ},I_{CQ})$和斜率$-\frac{1}{R_L}$画出交流负载线
			\end{itemize}
		\subsection{小信号模型分析法}
			\paragraph{基本思想}:输入信号变化范围很小时,可以认为BJT特性曲线基本为线性。即把非线性转化为线性的工程处理方法。
			\paragraph{模型}:双端口网络:\par
			输入特性方程:
			\[i_B=\]
			BE之间等效电阻$R_{BE}$电流方向由B到E, CE之间等效受控电流源,电流方向由C到E大小为$I_{BE}$
			
			
\end{document}