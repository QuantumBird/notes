\documentclass{beamer}

\usepackage[UTF8,space,hyperref]{ctex}
\usepackage{listing}
\usepackage{geometry}  
\usepackage{algorithm}  
\usepackage{algorithmicx}  
\usepackage{algpseudocode}  
\usepackage{amsmath} 
 
\usetheme{Madrid}
\author{高浚哲}
\institute{西安邮电大学}
\title{线性表}

\begin{document}
	
	\frame{\titlepage}
	%\frame{\tableofcontents[hideallsubsections]}
	
	\begin{frame}[c]\frametitle{线性表基本介绍}
		\textbf{线性表的定义:}
		\begin{block}{定义}
			一个线性表是$n$个具有相同特性的数据元素的有限序列。数据元素是一个抽象的符号,其具体含义在不同的情况下一般不同 \par
			线性表的相邻元素之间存在着序偶关系。如用$(a_1,\dots, a_{i - 1}, a_i, a_{i + 1},\dots,a_n)$ 表示一个顺序表,则表中$a_{i - 1}$领先于$a_i$,$a_i$领先于$a_{i+1}$,称$a_{i-1}$是$a_i$的直接前驱元素,$a_{i+1}$是$a_i$的直接后继元素。当$i=1,2,\dots,n-1$时,$a_i$有且仅有一个直接后继,当$i=2,3,\dots,n$时,$a_i$有且仅有一个直接前驱
		\end{block}
		%\\ \hspace*{\fill} \\
		\textbf{线性表的实现主要有两种方式:}
		\begin{itemize}
			\item \textbf{顺序表}\ 所有的元素都是相邻的
			\item \textbf{链表}\ 元素与元素之间不一定相邻
		\end{itemize}
	\end{frame}
	
	\begin{frame}[fragile]\frametitle{顺序表的定义及接口}
		%\textbf{线性表的定义}
		\begin{block}{定义}
\begin{verbatim}
template<typename T> class List{
    T * lst;   // 指向顺序表的指针
    int len, maxlen;  //当前表的长度;当前表的最大长度
public:
    List();    // 顺序表的构造函数:构造一个顺序表
    ~List();   // 顺序表的析构函数:销毁一个顺序表
    T * begin(){return &(lst[0]);}  // 返回指向表头的指针
    T * end(){return lst + len;}    // 返回指向表尾的指针
    T & operator[](int idx);   // 顺序表的随机访问函数
    void push_back(T val);     // 将一个元素插入顺序表尾部
    int size();   // 返回顺序表的大小
    T * find(const T & val);   // 通过值来查找表中元素的位置
    void insert(T * pos, T val); // 将元素插入到表中指定位置
    void remove(T * pos);    // 删除顺序表指定位置处的元素
};
\end{verbatim}
		\end{block}
	\end{frame}

	\begin{frame}[fragile]\frametitle{顺序表的构造和销毁}
		\textbf{顺序表的构造}
		\begin{block}{构造}
\begin{verbatim}
template<typename T>
List<T>::List(){
    lst = NULL; len = 0; maxlen = 4; // 初始化表的参数
    assert(allocater() == 0); // 为表分配内存并处理错误
}
\end{verbatim}
		\end{block}
		\ \ \ \ 构造一个顺序表的所有操作时间都是已知的,故构造顺序表的时间复杂度是$\mathcal{O}(1)$。\\
		\ \ \ \ 	其中 allocater() 实现了表的内存分配,每调用一次可以使顺序表的空间翻倍。\\
		\textbf{顺序表的销毁}\\
		\ \ \ \ 销毁顺序表的时候,释放顺序表所占的内存。
	\end{frame}

	\begin{frame}[fragile]\frametitle{顺序表的随机访问}
		\begin{block}{随机访问}
\begin{verbatim}
template<typename T>
T & List<T>::operator[](int idx){
    assert(0 <= idx && idx < len);  // 检查idx处是否有元素
    return lst[idx];   // 返回顺序表idx处的元素
}
\end{verbatim}
		\end{block}
		\ \ \ \ 顺序表是一种限制元素的地址是连续的的线性表。由于这个限制,我们可以通过一次加法运算得到顺序表内任意元素的地址(顺序表首地址+偏移量),故顺序表访问任意元素的时间复杂度是$\mathcal{O}(1)$。
	\end{frame}

	\begin{frame}[fragile]\frametitle{顺序表的插入}
		\begin{block}{将元素插入尾部}
\begin{verbatim}
template<typename T>
void List<T>::push_back(T val){
    if(len == maxlen)    // 检查顺序表是否已满
        assert(allocater() == 0);
    lst[len ++] = val;    // 将元素插入顺序表的尾部
}
\end{verbatim}
		\end{block}
		\ \ \ \ 由于顺序表中的元素是顺序排列的,故将元素插入尾部的时间复杂度是$\mathcal{O}(1)$(当顺序表长度大于最大长度,进行扩张时,时间复杂度为$\mathcal{O}(n)$ (其中$n$为表的长度))
	\end{frame}

	\begin{frame}[fragile]\frametitle{顺序表的插入}
		\begin{block}{将元素插入指定位置}
\begin{verbatim}
template<typename T>
void List<T>::insert(T * pos, T val){
    assert(pos >= begin() && pos <= end());  // 越界检查
    if(len == maxlen)  // 若表已满,则扩张表
        assert(allocater() == 0);
    for(T * iter = end(); iter != pos; iter --)
        *iter = *(iter - 1);  // 将指定位置前的元素后移一位
    *pos = val;  // 将元素插入指定位置
    len ++;  // 更新长度
}
\end{verbatim}
		\end{block}
		\ \ \ \ 在顺序表中,将一个元素插入指定位置,需要移动后面的所有元素,故其时间复杂度为$\mathcal{O}(n)$。
	\end{frame}

	\begin{frame}[fragile]\frametitle{顺序表的删除}
		\begin{block}{删除指定位置的元素}
\begin{verbatim}
template<typename T>
void List<T>::remove(T * pos){
    assert(begin() <= pos && pos < end());  //越界检查
    for(T * iter = pos; iter + 1 != end(); iter ++)
        *iter = *(iter + 1);  // 将元素向前移一位
    len --;  // 更新长度
}
\end{verbatim}
		\end{block}
	\ \ \ \ 与插入操作类似,在顺序表中删除操作需要移动指定位置后的所有元素,故其时间复杂度同为$\mathcal{O}(n)$。
	\end{frame}

	\begin{frame}[fragile]\frametitle{顺序表的查找}
		\begin{block}{查找}
\begin{verbatim}
template<typename T>
T * List<T>::find(const T & val){
    for(int i = 0; i < len; i ++){   // 线性查找
        if(lst[i] == val)
            return &(lst[i]);  // 若找到,则返回元素地址
    }
    return NULL;    // 若未找到,则返回 NULL
}
\end{verbatim}
		\end{block}
	\ \ \ \ 同样,在顺序表中查找时,通常的做法是便利顺序表,若找到指定的元素则返回,故其时间复杂度为$\mathcal{O}(n)$。
	\end{frame}

	\begin{frame}\frametitle{顺序表总结}
		\textbf{顺序表主要操作的时间复杂度:}
		\begin{itemize}
			\item \textbf{getitem}\ \ \ \  随机访问 $\mathcal{O}(1)$
			\item \textbf{find} \ \ \ \ \ \ \ \ 查找指定元素 $\mathcal{O}(n)$
			\item \textbf{insert} \ \ \ \ \ \ 插入元素到指定位置 $\mathcal{O}(n)$
			\item \textbf{remove} \ \ \ 移除指定位置处的元素 $\mathcal{O}(n)$
			\item \textbf{push\_back} 在表尾插入元素 $\mathcal{O}(1)$
		\end{itemize} 
		%\\ \hspace*{\fill} \\
		\textbf{顺序表适用的场景:} \\
		\ \ \ \ 由于顺序表进行随机访问的代价很小,故顺序表适用于进行频繁随机访问的场景(如将连续的字符映射成不同的元素,或存储密集的数据(如稠密的矩阵))
	\end{frame}
	\begin{frame}\frametitle{思考题}
		\begin{block}{思考题}
			\ \ \ \ 在插入排序算法中,我们使用一种线性查找来(反向)扫描已排好的子序列$A[1,\dots, j-1]$,我们可以使用二分查找来把插入排序的最坏情况总运行时间改进到$\Theta(nlgn)$吗?
		\end{block}
	\end{frame}
\end{document}