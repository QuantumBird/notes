\documentclass{beamer}

\usepackage[UTF8,space,hyperref]{ctex}
\usepackage{listing}
\usepackage{geometry}  
\usepackage{algorithm}  
\usepackage{algorithmicx}  
\usepackage{algpseudocode}  
\usepackage{amsmath} 
 
\usetheme{Madrid}
\author{高浚哲}
\institute{西安邮电大学}
\title{线性表}

\begin{document}
	
	\frame{\titlepage}
	%\frame{\tableofcontents[hideallsubsections]}
	
	\begin{frame}[c]\frametitle{线性表基本介绍}
		\textbf{线性表的定义:}
		\begin{block}{定义}
			一个线性表是$n$个具有相同特性的数据元素的有限序列。数据元素是一个抽象的符号,其具体含义在不同的情况下一般不同 \par
			线性表的相邻元素之间存在着序偶关系。如用$(a_1,\dots, a_{i - 1}, a_i, a_{i + 1},\dots,a_n)$ 表示一个顺序表,则表中$a_{i - 1}$领先于$a_i$,$a_i$领先于$a_{i+1}$,称$a_{i-1}$是$a_i$的直接前驱元素,$a_{i+1}$是$a_i$的直接后继元素。当$i=1,2,\dots,n-1$时,$a_i$有且仅有一个直接后继,当$i=2,3,\dots,n$时,$a_i$有且仅有一个直接前驱
		\end{block}
		%\\ \hspace*{\fill} \\
		\textbf{线性表的实现主要有两种实现方式:}
		\begin{itemize}
			\item 顺序表
			\item 链表
		\end{itemize}
	\end{frame}
	
	\begin{frame}[fragile]\frametitle{线性表的接口}
		\textbf{线性表的接口}
		\begin{block}{接口}
			\textbf{newList}()-> LinearList\\
			\textbf{getItem}(list: LinearList, idx: Int) -> ListElem \\
			\textbf{len}(list: LinearList) -> Int
		\end{block}
	\end{frame}

	\begin{frame}[fragile]\frametitle{Your title}
\begin{verbatim}
int add(int a, int b){
    return a + b;
}
\end{verbatim}
	\end{frame}


\end{document}