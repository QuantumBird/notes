\documentclass[UTF8]{ctexart}
\usepackage{amsmath}
\title{概率论与数理统计笔记}
\author{QuantumBird}
\date{\today}
\begin{document}
	\maketitle
	\tableofcontents
	\section{随机事件与概率}
	本章主要讲述了随机事件与概率的定义,为下面的概率论学习打下基础
	\subsection{随机试验\ 样本空间}
		\paragraph{确定现象}
		确定现象指的是在条件一定时,其结果也是一定的,这样的事件被称之为确定现象。
		\paragraph{随机现象}
		随机现象指的是即使在条件一定时,结果也不是确定的,但在大量重复的试验下,又呈现出一定的规律性的现象,其中,随机现象中的规律被称之为统计规律。
		%\paragraph{疑问}

		疑问:随机现象之中一定蕴含统计规律么?换言之,是否存在没有任何规律的现象?如果存在,那么它是随机现象么?
		\subsubsection{随机试验}
		\paragraph{试验} 对某一事物的某一特征的一次观察,测量或进行一次科学实验等
		
		这里的实验不一定是真实存在的,只要满足一定条件的行为都可以称之为试验。
		\paragraph{随机试验}一般地,如果一个试验满足下列条件:
		\begin{enumerate}
			\item 在相同的条件下可以重复进行
			\item 每次实验的结果不止一个,并且在试验前就可以明确所有结果
			\item 进行一次实验之前不能预知出现的结果
		\end{enumerate}
		称这样的试验为\textbf{随机试验},用$E$表示。
		
		注:这里的随机事件的定义为老师在讲课时给出的定义,教材上随机事件的原定义不包含第一条。包含第一条性质的随机试验被称为\textbf{可重复的随机试验},但不可重复的随机试验超出了教材的讨论范围,故暂时将可重复的随机事件称为随机事件。
		\subsubsection{样本空间}
		
\end{document}
