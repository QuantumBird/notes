\documentclass[UTF8]{ctexart}
\usepackage{amsmath}
\usepackage{mathrsfs}
\title{概率论与数理统计笔记}
\author{QuantumBird}
\date{\today}
\begin{document}
	\maketitle
	\tableofcontents
	\section{随机事件与概率}
	本章主要讲述了随机事件与概率的定义,为下面的概率论学习打下基础
	\subsection{随机试验\ 样本空间}
		\paragraph{确定现象}
		确定现象指的是在条件一定时,其结果也是一定的,这样的事件被称之为确定现象。
		\paragraph{随机现象}
		随机现象指的是即使在条件一定时,结果也不是确定的,但在大量重复的试验下,又呈现出一定的规律性的现象,其中,随机现象中的规律被称之为统计规律。
		%\paragraph{疑问}

		疑问:随机现象之中一定蕴含统计规律么?换言之,是否存在没有任何规律的现象?如果存在,那么它是随机现象么?
		\subsubsection{随机试验}
		\paragraph{试验} 对某一事物的某一特征的一次观察,测量或进行一次科学实验等
		
		这里的实验不一定是真实存在的,只要满足一定条件的行为都可以称之为试验。
		\paragraph{随机试验}一般地,如果一个试验满足下列条件:
		\begin{enumerate}
			\item 在相同的条件下可以重复进行
			\item 每次实验的结果不止一个,并且在试验前就可以明确所有结果
			\item 进行一次实验之前不能预知出现的结果
		\end{enumerate}
		称这样的试验为\textbf{随机试验},用$E$表示。
		
		注:这里的随机事件的定义为老师在讲课时给出的定义,教材上随机事件的原定义不包含第一条。包含第一条性质的随机试验被称为\textbf{可重复的随机试验},但不可重复的随机试验超出了教材的讨论范围,故暂时将可重复的随机事件称为随机事件。
		\subsubsection{样本空间}
		对于一个随机试验,我们虽无法得知其在某条件下的具体结果,但我们可以得知一个随机试验所有可能的结果(随机试验\ 条件2),则我们可以定义\textbf{样本空间}和\textbf{样本点}如下:
		\paragraph{样本空间} 随机试验$E$的所有可能结果组成的集合称为$E$的样本空间。记为$S$。
		\paragraph{样本点} $E$中的每个结果,即样本空间$S$中的每个元素,称为样本点。
	\subsection{随机事件}
		\subsubsection{随机事件}
		\paragraph{随机事件} 一般地,称试验$E$的样本空间$S$的子集为$E$的随机事件,简称为事件。
		\paragraph{基本事件}由一个样本点组成的单点集称为基本事件。
		\paragraph{必然事件}样本空间$S$自身被称为必然事件,因为它在每次随机试验中是一定发生的。
		\paragraph{不可能事件}空集$\emptyset \subset S$,且不包含任何样本点,在试验中不发生,称为不可能事件。
		\subsection{事件间的关系和运算}
		由前文的定义可知,随机事件本质是一个集合,故其具有集合的所有运算,部分运算在概率论中有一定的意义,下面从事件的角度上定义事件之间的关系和运算。
		
		设试验$E$的样本空间为$S$,而$A,B,A_k(k=1,2,\dots)$是$S$的子集,则我们可以定义如下的关系和运算:
		\begin{enumerate}
			\item \textbf{事件的包含} \ 若事件$A$的发生会导致事件$B$的发生,则称事件$B$包含事件$A$,记为$B \supset A$或$A \subset B$。
			\item \textbf{事件的相等} \ 若有$A \subset B$且$B \subset A$,则$A$与$B$相等,记为$A = B$。
			\item \textbf{和事件} \ 事件$A$与$B$至少有一个发生的事件,称为事件$A$与$B$的和事件,记为$A \cup B$,或$A + B$。
			% TODO积事件等。。。。。
 		\end{enumerate}
		
\end{document}
